\section{Background and Current State of Indexing on Sui}

To understand the challenges and opportunities in Sui blockchain data indexing, it is essential to first establish a clear understanding of the Sui architecture, data model, and current indexing approaches.

\subsection{Sui Blockchain Architecture}

Sui is a high-performance Layer 1 blockchain designed with an object-centric data model, which differentiates it from account-based blockchains like Ethereum. In Sui, the fundamental unit of storage is an object rather than an account, and each object has a unique identifier (ID) that persists throughout its lifetime.

Key architectural elements of Sui include:

\begin{itemize}
    \item \textbf{Object-Centric Model:} All on-chain state is represented as objects that can be created, modified, or deleted through transactions.
    \item \textbf{Move Programming Language:} Sui uses Move, a safe and expressive programming language designed for digital assets.
    \item \textbf{Parallel Transaction Execution:} Sui can execute non-conflicting transactions in parallel, significantly increasing throughput.
    \item \textbf{Checkpoint-Based Consensus:} Instead of traditional blocks, Sui uses checkpoints as units of finality, which contain batches of transaction certificates.
\end{itemize}

This architecture offers exceptional performance but creates unique challenges for data indexing compared to more traditional blockchain architectures.

\subsection{Data Flow and Transaction Lifecycle in Sui}

\subsubsection{Transaction Structure}

A Sui transaction typically follows this lifecycle:

\begin{enumerate}
    \item \textbf{Transaction Creation:} A user creates a transaction specifying objects to be read, modified, or created.
    \item \textbf{Transaction Signing:} The transaction is signed by the sender and potentially other stakeholders.
    \item \textbf{Transaction Submission:} The signed transaction is submitted to validators.
    \item \textbf{Execution:} Validators execute the transaction and produce effects.
    \item \textbf{Finalization:} The transaction is finalized in a checkpoint.
\end{enumerate}

Each transaction in Sui produces a comprehensive set of data:

\begin{itemize}
    \item \textbf{Transaction Data:} Contains the sender, gas payment, and programmable transaction commands.
    \item \textbf{Transaction Effects:} Records created, modified, and deleted objects.
    \item \textbf{Events:} Custom events emitted during transaction execution.
    \item \textbf{Object Changes:} Detailed information about object state changes.
\end{itemize}

\subsubsection{Checkpoint Structure}

Checkpoints are the fundamental unit of data organization in Sui. Each checkpoint contains:

\begin{itemize}
    \item A sequence number
    \item A set of transaction digests
    \item A timestamp
    \item Network metadata
\end{itemize}

This structure is crucial for indexers as they typically process data on a checkpoint-by-checkpoint basis.

\subsection{Current Indexing Methods and Schemas}

\subsubsection{Native Sui Indexing}

The Sui framework provides a basic indexing mechanism through its Full Node API, which allows querying for:

\begin{itemize}
    \item Transactions by digest
    \item Objects by ID
    \item Events by type
    \item Recent checkpoints
\end{itemize}

However, this approach has significant limitations:

\begin{itemize}
    \item \textbf{Query Restrictions:} Limited ability to perform complex or custom queries
    \item \textbf{Retention Policy:} Historical data may not be indefinitely available
    \item \textbf{Performance:} Not optimized for analytics or high-frequency queries
    \item \textbf{Lack of Custom Indexing:} No easy way to index only specific data relevant to an application
\end{itemize}

\subsubsection{Current Third-Party Indexing Solutions}

To overcome the limitations of native indexing, several third-party solutions have emerged:

\begin{itemize}
    \item \textbf{Sentio:} Provides an SDK for creating custom indexers but requires using their cloud infrastructure.
    \item \textbf{SubQuery:} Offers a general blockchain indexing solution with Sui support.
    \item \textbf{Custom Rust Indexers:} Many teams build fully custom indexers using the Sui Rust SDK.
\end{itemize}

\subsubsection{Typical Indexing Schema}

Most indexing solutions for Sui follow a similar schema pattern:

\begin{itemize}
    \item \textbf{Transactions Table:} Stores transaction metadata and context
    \item \textbf{Effects Table:} Records transaction effects and state changes
    \item \textbf{Events Table:} Captures custom events emitted during execution
    \item \textbf{Objects Table:} Tracks object creation, modification, and deletion
    \item \textbf{Checkpoints Table:} Maintains checkpoint metadata for synchronization
\end{itemize}

This schema design generally works well but requires significant expertise to implement and maintain.

\subsection{Limitations and Challenges of Current Approaches}

Current indexing approaches on Sui face several significant challenges:

\begin{itemize}
    \item \textbf{Technical Complexity:} Implementing a custom indexer requires deep understanding of the Sui framework, Rust programming, and database design.
    
    \item \textbf{Resource Intensity:} Indexing solutions typically require dedicated developers and ongoing maintenance, diverting resources from core application development.
    
    \item \textbf{Data Sovereignty Issues:} Third-party solutions often require teams to surrender control over their data, creating potential vendor lock-in and privacy concerns.
    
    \item \textbf{Limited Customizability:} Most existing solutions either offer too little flexibility or require teams to build everything from scratch.
    
    \item \textbf{Synchronization Challenges:} Maintaining synchronization with the blockchain while handling forks and reorganizations is complex.
    
    \item \textbf{Scalability Concerns:} As applications grow, indexing needs can change dramatically, requiring solutions that can scale efficiently.
\end{itemize}

The considerable expertise required to implement effective indexing solutions creates a significant barrier to entry for many teams. This barrier is particularly problematic for teams without dedicated blockchain specialists, effectively limiting innovation and adoption in the Sui ecosystem.

\subsection{The Developer Experience Gap}

Our research with Sui developers and the DevRel team has identified a significant experience gap in the current ecosystem. The lack of accessible, self-custodial indexing solutions has created a situation where:

\begin{itemize}
    \item Teams must either invest heavily in technical expertise or rely on third-party services.
    \item Approximately 30\% of development time is dedicated to indexing-related tasks.
    \item Many projects compromise on data access capabilities due to resource constraints.
    \item Analytics and dashboard creation becomes a significant challenge.
\end{itemize}

This experience gap represents not just a technical challenge but a strategic opportunity to improve the developer experience on Sui. The following sections will detail our approach to addressing these challenges through two complementary contributions. 